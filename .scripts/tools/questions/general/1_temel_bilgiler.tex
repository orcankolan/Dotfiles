\begin{question}[subtitle=]
Penetrant muayenesi hangi tür malzemelere uygulanamaz?
	\begin{tasks}
          \task Titanyum malzemeler 
          \task Ferromanyetik malzemeler 
          \task Aşırı gözenekli malzemeler \correct
          \task b ve c şıkları
	\end{tasks}
\end{question}
\begin{solution}
	\correct
\end{solution}

\begin{question}[subtitle=]
Aşağıdakilerden hangisi penetrant türlerinden veya tiplerinden değildir?
	\begin{tasks}
          \task Su bazlı \correct
          \task Sonradan emülsiyon haline gelebilir tür
          \task Florışıl
          \task Tip III (Çift Amaçlı) tip
	\end{tasks}
\end{question}
\begin{solution}
	\correct
\end{solution}

\begin{question}[subtitle=]
	Penetrant muayenesi metodu ile ilgili aşağıdakilerden hangisi doğrudur?
	
	\begin{tasks}
		\task Hemen her tür malzemeye uygulanabilir
		\task Yüzeye açık hataların saptanması için uygundur
		\task Penetrant sistemi test parçası ile reaksiyona girmiyor ise uygundur
		\task Yukarıdakilerin hepsi \correct
	\end{tasks}
\end{question}
\begin{solution}
	\correct
\end{solution}

\begin{question}[subtitle=]
	Aşağıdaki ifadelerden hangisi ya da hangileri doğrudur?
	
	\begin{tasks}
		\task Penetrant metodu malzeme yüzeyine yakın süreksizliklerin görünür hale getirildiği bir tahribatsız muayene metodudur
		\task Muayene parçalarının yüzey durumlarının penetrant testine bir etkisi yoktur
		\task Boya penetrant kullanıldığında hata belirtisi koyu mor zemin üzerinde sarı renkle ortaya çıkar
		\task Penetrant muayenesi yalnızca malzeme yüzeyine açılan süreksizliklerin görünür hale
getirildiği bir yöntemdir \correct
	\end{tasks}
\end{question}
\begin{solution}
	\correct
\end{solution}

\begin{question}[subtitle=]
	İş ve İşçi sağlığı güvenliğinden kim sorumludur?
	\begin{tasks}
          \task Tahribatsız muayene operatörü
          \task İSG uzmanı
          \task İSG denetleyicileri
          \task Tüm çalışanlar \correct
	\end{tasks}
\end{question}
\begin{solution}
	\correct
\end{solution}

\begin{question}[subtitle=]
  Penetrant muayenesi için aşağıdakilerden hangisi güvenlik tedbiri gerektirmez?
	\begin{tasks}
          \task Mor ötesi ışınım
          \task 2000lx-3000lx arası beyaz ışık \correct
          \task Hava kirliliği
          \task Deri tahrişi
	\end{tasks}
\end{question}
\begin{solution}
	\correct
\end{solution}


\begin{question}[subtitle=]
  Aşağıdakilerden hangisi Penetrant Muayenesindeki güvenlik tedbirleri için doğru bir ifadedir?
	\begin{tasks}
          \task Florışıl penetrantlar ile yapılan muayenelerde kullanılan mor öresi ışınım dalga boyu 320nm den küçük olmalıdır.
          \task Açık tanklarda bulunan penetrant malzemelerinin parlama noktası yüksek olmalıdır. \correct
          \task Yağ bazlı penetrantlar cildin aşırı yağlanmasına sebep olur.
          \task Basınçlı tüplerde bulunan penetrantların alev alma ihtimali yoktur.
	\end{tasks}
\end{question}
\begin{solution}
	\correct
\end{solution}


\begin{question}[subtitle=]
  	\begin{figure}[!htb]
		\centering
		\fbox{\includegraphics[width=0.35\textwidth]{ww4}}
	\end{figure}

  Resimde verilen uyarı resminin tanımı aşağıdakilerden hangisidir?
	\begin{tasks}
          \task Zehirli madde tehlikesi
          \task Biyolojik tehlike
          \task Korozif tehlike \correct
          \task Metan tehlikesi
	\end{tasks}
\end{question}
\begin{solution}
	\correct
\end{solution}

\begin{question}[subtitle=]
  	\begin{figure}[!htb]
		\centering
		\fbox{\includegraphics[width=0.35\textwidth]{ww23}}
	\end{figure}

  Resimde verilen uyarı resminin tanımı aşağıdakilerden hangisidir?
	\begin{tasks}
          \task Yüksek voltaj tehlikesi \correct
          \task Biyolojik tehlike
          \task Korozif tehlike 
          \task Zehirli madde tehlikesi
	\end{tasks}
\end{question}
\begin{solution}
	\correct
\end{solution}


