\begin{question}[subtitle=]
Penetrant muayenesi ürün ailesini aşağıdakilerden hangisi tanımlar?
	\begin{tasks}
          \task Penetrant muayenesi için gerekli tüm cihaz ve donanım
          \task Florışıl, Boya ve Florışıl-Boya penetrant sistemi
          \task Penetrant, ara temizleyici ve geliştirici \correct
          \task Boya penetrant muayenesi için penetrant, geliştirici ve güçlü ışık kaynağı; Florışıl penetrant muayenesi için ise penetrant, geliştirici ve UV Lambası
	\end{tasks}
\end{question}
\begin{solution}
	\correct
\end{solution}

\begin{question}[subtitle=]
  Penetrant muayenesi öncesinde yapılan görsel muayene esnasında muayene parçası
  üzerinde çatlak olmayan ancak derin çizikler gözlenmektedir. Penetrant muayenesi
  sonucunda bu bölgelerde ne tür belirtiler beklenir?
	\begin{tasks}
          \task İlgili belirtiler
          \task Yalnış belirtiler
          \task İlgizisiz belirtiler \correct
          \task Hata belirtileri
	\end{tasks}
\end{question}
\begin{solution}
	\correct
\end{solution}

\begin{question}[subtitle=]
  Penetrant muayenesi ara yıkama sırasında kullanılan bezlerin geride bıraktığı lifler
  yüzeyde ne tür belirtilerin oluşmasına sebep olacaktır?
	\begin{tasks}
          \task İlgili belirtiler
          \task Yalnış belirtiler \correct
          \task İlgizisiz belirtiler 
          \task Hata belirtileri
	\end{tasks}
\end{question}
\begin{solution}
	\correct
\end{solution}

\begin{question}[subtitle=]
  Penetrant muayenesi öncesi yapılan kimyasal dağlama hangi amaç(lar) ile yapılabilir?
	\begin{tasks}
          \task Pas giderme
          \task Kusur içini dolduran metalik kalıntıları giderme
          \task Organik kalıntıları giderme
          \task a ve b şıkları \correct
	\end{tasks}
\end{question}
\begin{solution}
	\correct
\end{solution}

\begin{question}[subtitle=]
  UV-metre ile ölçülen 500 \si[per-mode=symbol]{\micro\watt\per\centi\m^{2}} kaç \si[per-mode=symbol]
{\watt\per\m^{2}} eder?
	\begin{tasks}
          \task 50
          \task 5 \correct
          \task 5000
          \task 5
	\end{tasks}
\end{question}
\begin{solution}
	\correct
\end{solution}

\begin{question}[subtitle=]
  Lipofilik (yağ bazlı) emülsiyon yapıcılar için aşağıdaki ifadelerden hagisi(leri) yalnıştır?
	\begin{tasks}
          \task Fırça ile tatbik edilirler \correct
          \task Difüzyon mekanizması ile penetranta etki ederler
          \task Emülsiyonlaştırma süreleri kısa ve kritiktir
          \task Doğrudan kullanıma hazırdırlar (çözelti hazırlamaya gerek yoktur)
	\end{tasks}
\end{question}
\begin{solution}
	\correct
\end{solution}

\begin{question}[subtitle=]
  Aşağıdaki muayene parçası malzemelerinden hangisinde en fazla penetrasyon süresi beklenir?
	\begin{tasks}
          \task Alüminyum
          \task Östenitik paslanmaz çelik \correct
          \task Çelik
          \task Bakır
	\end{tasks}
\end{question}
\begin{solution}
	\correct
\end{solution}

\begin{question}[subtitle=]
  Florışıl penetrant ile yapılan bir muayenede su ile ara yıkama ile ilgili aşağıdakilerden
  hangisi(leri) doğrudur?
	\begin{tasks}
          \task Su ile yıkama 90 derece açı ile yapılmalıdır
          \task Yıkama en az 2 \si[per-mode=symbol] {\watt\per\m^{2}}'lık bir aydınlatma ile yapılmalıdır
          \task Geliştirici olarak kuru toz kullanılacak ise ara yıkama sonrası kurutma gereklidir \correct
          \task Yukarıdakilerin hepsi
	\end{tasks}
\end{question}
\begin{solution}
	\correct
\end{solution}

\begin{question}[subtitle=]
      	\begin{figure}[!htb]
		\centering
		\fbox{\includegraphics[width=0.75\textwidth]{uvlxmetre}}
	\end{figure}

  Yukarıda verilen resimdeki cihazda okunan değerler göz önüne alındığında aşağıdakilerden
  hangisi(leri) yalnıştır?
	\begin{tasks}
          \task 2.634 \si[per-mode=symbol] {\watt\per\m^{2}} aydınlatma okunmaktadır
          \task Florışıl penetrant ile muayene ortam koşulları sağlanmamaktadır
          \task 26.34 \si[per-mode=symbol] {\watt\per\m^{2}} aydınlatma okunmaktadır
          \task a ve b şıkları \correct
	\end{tasks}
\end{question}
\begin{solution}
	\correct
\end{solution}

\begin{question}[subtitle=]
  Bir penetrant muayenesi için IDa-4 ürün ailesi kullanılmaktadır. Buna göre
  aşağıdaki ifadelerden hagisi(leri) doğrudur?
	\begin{tasks}
          \task Renkli penetrant kullanılmaktadır
          \task Kuru toz geliştirici kullanılmaktadır
          \task Emülsiyon yapıcı tatbiki öncesi su ile ön temizlik yapılır
          \task b ve c şıkları \correct
	\end{tasks}
\end{question}
\begin{solution}
	\correct
\end{solution}

\begin{question}[subtitle=]
	Penetrant muayenesi ürün ailesini aşağıdakilerden hangisi tanımlar?
	
	\begin{tasks}
		\task Penetrant muayenesi için gerekli tüm cihaz ve donanım
		\task Florışıl, Boya ve Florışıl-Boya penetrant sistemi
		\task Boya penetrant muayenesi için penetrant, geliştirici ve güçlü ışık kaynağı;
                Florışıl penetrant muayenesi için ise penetrant, geliştirici ve UV Lambası
		\task Penetrant, ara temizleyici ve geliştirici \correct
	\end{tasks}
\end{question}
\begin{solution}
	\correct
\end{solution}

\begin{question}[subtitle=]
  Penetrant muayenesinde ön temizleme sırasında malzeme sıcaklığı hangi aralıkta olmalıdır?
  \begin{tasks}
    \task \SI{50}{\degreeCelsius}'dan az
    \task \SI{10}{\degreeCelsius} - \SI{50}{\degreeCelsius} arasında
    \task Ön temizlik sırasında muayene parçası sıcaklığı önemli değildir \correct
    \task Curie sıcaklığının altında
  \end{tasks}
\end{question}
\begin{solution}
	\correct
\end{solution}

\begin{question}[subtitle=]
  Aşağıdakilerden hangisi mekanik ön temizleme tekniği değildir?
  \begin{tasks}
    \task Kumlama
    \task Dağlama \correct
    \task Tel fırça ile fırçalama
    \task Zımparalama
  \end{tasks}
\end{question}
\begin{solution}
	\correct
\end{solution}

\begin{question}[subtitle=]
  Mekanik ön temizlik aşağıdaki kalıntılardan hangisinin temizlenmesi için uygundur?
  \begin{tasks}
    \task Yağ
    \task Gres
    \task Tufal \correct
    \task Yukarıdakilerin hepsi
  \end{tasks}
\end{question}
\begin{solution}
	\correct
\end{solution}

\begin{question}[subtitle=]
  Test edilecek parça üzerinde pas olduğu tespit edilmiş ise hangi ön temizleme metodu uygundur?
  \begin{tasks}
    \task Aseton ile silme
    \task Su ile yıkama
    \task Pas muayeneyi etkilemez
    \task Mekanik temizlik sonrası kimyasal temizlik \correct
  \end{tasks}
\end{question}
\begin{solution}
	\correct
\end{solution}

\begin{question}[subtitle=]
  Penetrantın yüzeye tatbiki ile ara temizlemenin başlaması arasında geçen süreye ne ad verilir?
  \begin{tasks}
    \task Test bekleme süresi
    \task Geliştirme süresi
    \task Penetrasyon süresi \correct
    \task Emülsiyonlaştırma süresi
  \end{tasks}
\end{question}
\begin{solution}
	\correct
\end{solution}

\begin{question}[subtitle=]
  Penetrasyon süresi hakkında aşağıdakilerden hangisi doğrudur?
  \begin{tasks}
    \task 5 ila 60 dakika arasında değişir
    \task Penetrantın malzeme yüzeyinden hatanın en dip ucuna nüfuz etme süresidir
    \task Muayene parçası malzemesine göre değişir \correct
    \task Yukarıdakilerin hepsi 
  \end{tasks}
\end{question}
\begin{solution}
	\correct
\end{solution}

\begin{question}[subtitle=]
  Emülsiyon yapıcılar aşağıdaki hangi yöntemle yüzeye tatbik edilemez?
  \begin{tasks}
    \task Püskürtme
    \task Daldırma
    \task Fırça ile \correct
    \task Dökme
  \end{tasks}
\end{question}
\begin{solution}
	\correct
\end{solution}

\begin{question}[subtitle=]
  Penetrant Muayenesinde aşırı kalın geliştirici tabakasına niçin izin verilmez?
  \begin{tasks}
    \task Son temizlik güçleşir 
    \task Maliyet gereksiz artar
    \task Küçük süreksizlikler içindeki az miktardaki penetrant yüzeye ulaşmayabilir
    \task Yukarıdakilerin hepsi doğrudur \correct
  \end{tasks}
\end{question}
\begin{solution}
	\correct
\end{solution}

\begin{question}[subtitle=]
  Aşağıdakilerden hangisi muayene sonucu oluşan ilgisiz belirtiye örnektir?
  \begin{tasks}
    \task Silme bezlerinden kalma lifler
    \task Parmak izleri
    \task Parça üzerinde tasarımdan kaynaklı izler \correct
    \task a ve b şıkları
  \end{tasks}
\end{question}
\begin{solution}
	\correct
\end{solution}

\begin{question}[subtitle=]
  İnceleme sırasında UV ışık şiddeti nerede ölçülmelidir?
  \begin{tasks}
    \task UV ışık kaynağının 300mm önünde
    \task UV ışık kaynağı ile muayene parçası arasında
    \task UV ışık kaynağının hemen önünde
    \task Muayene parçasının üzerinde \correct
  \end{tasks}
\end{question}
\begin{solution}
	\correct
\end{solution}




