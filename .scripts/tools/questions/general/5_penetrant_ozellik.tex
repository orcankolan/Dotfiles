\begin{question}[subtitle=]
Aşağıdakilerden hangisi(leri) geliştiricilerin sahip olması gereken özelliklerden biri değildir?
	\begin{tasks}
          \task Beyaz renkli olmalıdır
          \task Düşük özkütleli olmalıdır
          \task Görünür ışığı iyi yansıtmalıdır \correct
          \task Yukarıdakilerin hepsi
	\end{tasks}
\end{question}
\begin{solution}
	\correct
\end{solution}

\begin{question}[subtitle=]
Geliştirici uygulaması ile ilgili aşağıdakilerden hangisi doğrudur?
	\begin{tasks}
          \task Kuru toz sadece florışıl penetrantlar ile kullanılır \correct
          \task Kuru toz uygulamadan önce kurutma gerekmez
          \task Yaş geliştiricileri uygulamadan önce kurutma mutlaka gereklidir
          \task a ve c şıkları
	\end{tasks}
\end{question}
\begin{solution}
	\correct
\end{solution}

\begin{question}[subtitle=]
  Hidrofilik (su bazlı) emülsiyon yapıcılar için aşağıdaki ifadelerden hagisi(leri) yalnıştır?
	\begin{tasks}
          \task Daldırma yöntemi ile uygulanabilirler
          \task Difüzyon mekanizması ile penetranta etki ederler \correct
          \task Su toleransları sonsuzdur
          \task Çözelti halinde kullanılırlar
	\end{tasks}
\end{question}
\begin{solution}
	\correct
\end{solution}

\begin{question}[subtitle=]
  Sıvıların yüzey gerilimleri ile ilgili aşağıdaki ifadelerden hangisi(leri) doğrudur?
	\begin{tasks}
          \task Birimi kuvvet/alan olarak ifade edilir
          \task Adhesiv kuvvetlerin bir göstergesidir
          \task Kohesiv kuvvetlerin bir göstergesidir \correct
          \task a ve c şıkları
	\end{tasks}
\end{question}
\begin{solution}
	\correct
\end{solution}

\begin{question}[subtitle=]
  Penetrant sıvısının viskozitesinin yüksek olması aşağıdakilerden hangisine sebep olur?
	\begin{tasks}
          \task Kapiler basıncının düşük olmasına
          \task Penetrasyon süresinin yüksek olmasına \correct
          \task Penetrasyon süresinin düşük olmasına
          \task a ve b şıkları
	\end{tasks}
\end{question}
\begin{solution}
	\correct
\end{solution}

\begin{question}[subtitle=]
      	\begin{figure}[!htb]
		\centering
		\fbox{\includegraphics[height=0.75\textwidth]{Capillar}}
	\end{figure}

  Resimdeki sıvı için aşağıdakilerden hangisi(leri) söylenebilir?
	\begin{tasks}
          \task Düşük ıslatma açısına sahiptir \correct
          \task Viskositesi yüksektir
          \task Viskositesi düşüktür
          \task a ve c şıkları
	\end{tasks}
\end{question}
\begin{solution}
	\correct
\end{solution}

\begin{question}[subtitle=]
  Aşağıdakilerden hangisi kapiler basıncın yüksek olmasını sağlar
	\begin{tasks}
          \task Dar süreksizlik açıklığı
          \task Düşük ıslatma açısı
          \task Düşük sıvı yüzey gerilimi
          \task Yukarıdakilerin hepsi \correct
	\end{tasks}
\end{question}
\begin{solution}
	\correct
\end{solution}

\begin{question}[subtitle=]
  Yüksek sıcaklığın penetrantlar üzerinde nasıl olumsuz etkileri vardır?
	\begin{tasks}
          \task Florışıl penetrantların termal dayanıklılığını azaltır
          \task Florışıl penetrantların florışıma parlaklığını arttırır
          \task Penetrantların uçucu taşıyıcı sıvı bileşenlerinin kaybolmasına ve
          penetrantın süreksizliklere ulaşmasının zorlaşmasına sebep olur
          \task a ve c şıkları doğrudur \correct
	\end{tasks}
\end{question}
\begin{solution}
	\correct
\end{solution}

\begin{question}[subtitle=]
  Elektromanyetik dalgalar için aşağıdaki ifadelerden hangisi(leri) doğrudur?
	\begin{tasks}
          \task Dalga boyları arttıkça enerjileri düşer \correct
          \task Dalga boyları arttıkça enerjileri artar
          \task Görünür ışık aralığı 380 - 700 \si{\micro\m} aralığındadır
          \task a ve c şıkları
	\end{tasks}
\end{question}
\begin{solution}
	\correct
\end{solution}

\begin{question}[subtitle=]
  Aktivite, viskozite ve su toleransı aşağıdakilerden hangisine etki eden 3 temel faktördür?
	\begin{tasks}
          \task Penetrant
          \task Geliştirici
          \task Lipofilik emülsiyon yapıcı \correct
          \task Hidrofilik emülsiyon yapıcı
	\end{tasks}
\end{question}
\begin{solution}
	\correct
\end{solution}

\begin{question}[subtitle=]
  Bir penetrant muayenesi için IBa-4 ürün ailesi kullanılmaktadır. Buna göre
  aşağıdaki ifadelerden hagisi yalnıştır?
	\begin{tasks}
          \task Florışıl penetrant kullanılmaktadır
          \task Lipofilik emülsiyon yapıcı kullanılmaktadır
          \task Hassasiyet seviyesi 4 olan bir ürün ailesidir
          \task Emülsiyon yapıcı için çözelti hazırlanmalıdır \correct
	\end{tasks}
\end{question}
\begin{solution}
	\correct
\end{solution}

\begin{question}[subtitle=]
  Aşağıdakilerden hangisi çözücü bazlı geliştiricilerin işlevlerinden biri değildir?
	\begin{tasks}
          \task Penetrantın ince film kalınlığını çözücü ile genişletmek
          \task Florışıma parlaklığını azaltmak \correct
          \task Penetrant sıvısı üzerinde ``Sıvı emme'' etkisi yaratmak 
          \task Penetrant sıvısı üzerinde ``Parçacık yüzeyine tutunma'' etkisi yaratmak
	\end{tasks}
\end{question}
\begin{solution}
	\correct
\end{solution}

\begin{question}[subtitle=]
  Bir sıvının katı maddelerin yüzeyini iyi ıslatabildiğinin göstergesi aşağıdakilerden hangisidir?
  \begin{tasks}
    \task Adhezyon kuvvetlerinin düşük olması
    \task Düşük ıslatma açısı \correct
    \task Kohezyon kuvvetlerinin yüksek olması
    \task Yüksek ıslatma açısı
  \end{tasks}
\end{question}
\begin{solution}
	\correct
\end{solution}

\begin{question}[subtitle=]
  Penetrantın ıslatma gücü aşağıdakilerden hangisine bağlıdır?
  \begin{tasks}
    \task Yüzey pürüzlülüğüne
    \task Yüzey temizliğine
    \task Muayene parçası malzemesine
    \task Yukarıdakilerin hepsi \correct
  \end{tasks}
\end{question}
\begin{solution}
	\correct
\end{solution}


\begin{question}[subtitle=]
  Aşağıdaki ara temizleme tekniklerinden hangisinde aşırı yıkama riski en yüksektir?
  \begin{tasks}
    \task Kuru bez ile ara temizleme
    \task Emülsiyon yapıcı ve su ile ara temizleme
    \task Su ile ara temizleme
    \task Sprey çözücü ile temizleme \correct
  \end{tasks}
\end{question}
\begin{solution}
	\correct
\end{solution}

\begin{question}[subtitle=]
  Aşağıdaki penetrantlardan hangisi çözücü ile temizlenemez?
  \begin{tasks}
    \task Tip I penetrantlar
    \task Tip II penetrantlar 
    \task Sonradan emülsiyon haline gelebilen penetrantlar
    \task Yukarıdakilerin hepsi temizlenebilir \correct
  \end{tasks}
\end{question}
\begin{solution}
	\correct
\end{solution}

\begin{question}[subtitle=]
  Penetrant muayenesinde süreksizliklerin tespit edilebilirliliği aşağıdaki faktörlerden hangisine bağlıdır?
  \begin{tasks}
    \task Hatanın yüzey açıklığının genişliğine
    \task Yüzey pürüzlülüğüne
    \task Muayene personeli bilgisi ve tecrübesine
    \task Yukarıdakilerin hepsi \correct
  \end{tasks}
\end{question}
\begin{solution}
	\correct
\end{solution}

\begin{question}[subtitle=]
  Penetrantın nüfuziyet hızını etkileyen en önemli özelliği aşağıdakilerden hangisidir?
  \begin{tasks}
    \task Penetrantın özgül ağırlığı
    \task Penetrantın viskozitesi \correct
    \task Penetrantın yüzey gerilimi
    \task Yukarıdakilerin hepsi
  \end{tasks}
\end{question}
\begin{solution}
	\correct
\end{solution}

\begin{question}[subtitle=]
  Derinliği fazla olmayan ve geniş süreksizliklerin tespiti için en uygun penetrant sistemi
aşağıdakilerden hangisidir?
  \begin{tasks}
    \task Sonradan emülsiyon haline getirilebilir florışıl penetrant \correct
    \task Su ile temizlenebilir florışıl penetrant
    \task Çözücü ile temizlenebilir boya penetrant
    \task Su ile temizlenebilir boya penetrant
  \end{tasks}
\end{question}
\begin{solution}
	\correct
\end{solution}

\begin{question}[subtitle=]
  Bir sıvı içinde ince ve dağılmış halde asıltı olarak bulunan katı parçacıklar nasıl tanımlanır?
  \begin{tasks}
    \task Çözelti
    \task Karışım
    \task Emülsiyon
    \task Süspansiyon \correct
  \end{tasks}
\end{question}
\begin{solution}
	\correct
\end{solution}


