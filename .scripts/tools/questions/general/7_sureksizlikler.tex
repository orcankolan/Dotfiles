\begin{question}[subtitle=]
  Aşağıdakilerden hangisi ``içkin'' süreksizlikler sınıfına girmez?
	\begin{tasks}
          \task Sıcak yırtılma
          \task Çekinti boşluğu
          \task Yetersiz ergime \correct
          \task Gözenek
	\end{tasks}
\end{question}
\begin{solution}
	\correct
\end{solution}

\begin{question}[subtitle=]
  Aşağıdakilerden hangisi ``proses'' süreksizlikleri sınıfına girmez?
	\begin{tasks}
          \task Isıl işlem çatlağı
          \task Yorulma çatlağı \correct
          \task Laminasyon
          \task a ve b şıkları
	\end{tasks}
\end{question}
\begin{solution}
	\correct
\end{solution}

\begin{question}[subtitle=]
  Aşağıdaki ifadelerden hangisi kaynak dikişinde oluşabilecek çatlaklar için doğrudur?
	\begin{tasks}
          \task Krater çatlakları genellikle kaynak dikişi ortalarında meydana gelir
          \task Boyuna çatlakların kök bölgesinde görülme ihtimali (kepe kıyasla) daha fazladır
          \task Enine çatlaklar hidrojen gevrekliği sebebi ile ortaya çıkabilir \correct
          \task Boyuna çatlaklar ısıdan etkilenen bölgede oluşmazlar
	\end{tasks}
\end{question}
\begin{solution}
	\correct
\end{solution}

\begin{question}[subtitle=]
  Aşağıdakilerden hangisi ``servis'' süreksizlikleri sınıfına girmez?
	\begin{tasks}
          \task Korozyonlu gerilim çatlakları 
          \task Sıcak yırtılma \correct
          \task Yorulma çatlağı
          \task Termal şok çatlakları
	\end{tasks}
\end{question}
\begin{solution}
	\correct
\end{solution}

\begin{question}[subtitle=]
  Aşağıdakilerden hangisi penetrant muayenesi ile tespit edilemez?
	\begin{tasks}
          \task Laminasyon
          \task X kaynak ağızlı birleştirmede yetersiz nüfuziyet \correct
          \task Çekinti boşlukları
          \task Kaynaklı birleştirmelerde gözenek
	\end{tasks}
\end{question}
\begin{solution}
	\correct
\end{solution}

\begin{question}[subtitle=]
    	\begin{figure}[!htb]
		\centering
		\fbox{\includegraphics[width=0.40\textwidth]{Coldshut2}}
	\end{figure}

        Yukarıdaki resimde hangi süreksizlik gözlenmektedir?
	\begin{tasks}
          \task Damar
          \task Katlanma
          \task Döküm çatlağı
          \task Soğuk birleşme \correct
	\end{tasks}
\end{question}
\begin{solution}
	\correct
\end{solution}

\begin{question}[subtitle=]
  Aşağıdakilerden hangisi penetrant muayenesinde sınıflandırılan ``yuvarlak belirtiler'' ile ilgili doğru
  bir ifadedir?
	\begin{tasks}
          \task Boyları enlerinin 4 katından büyüktür
          \task Dövme ve haddeleme işlemlerinde görülmezler \correct
          \task Boyları enlerine eşittir
          \task Servis süreksizlikleri sınıfına girerler
	\end{tasks}
\end{question}
\begin{solution}
	\correct
\end{solution}

\begin{question}[subtitle=]
  İnceleme sadece geliştirme süresi sonunda yapılmış ise aşağıdakilerden hangisi doğrudur?
  \begin{tasks}
    \task Uygun inceleme koşullarında yapılmış ise doğrudur 
    \task Belirtinin sınıfı yanlış belirlenebilir \correct
    \task Süreksizliğin derinliği bulunabilir
    \task b ve c şıkları doğrudur
  \end{tasks}
\end{question}
\begin{solution}
	\correct
\end{solution}

\begin{question}[subtitle=]
  Penetrant testi sonucunda saptanan hatalar hakkında hangi bilgiler elde edilir?
  \begin{tasks}
    \task Hataların gerçek uzunluğu
    \task Hataların gerçek derinliği
    \task Hataların parça üzerindeki konumları \correct
    \task Yukarıdakilerin hepsi
  \end{tasks}
\end{question}
\begin{solution}
	\correct
\end{solution}

\begin{question}[subtitle=]
    	\begin{figure}[!htb]
		\centering
		\fbox{\includegraphics[width=0.40\textwidth]{FatigueCrackGear}}
	\end{figure}

        Yukarıdaki resimle ilgi aşağıdakilerden hangisi doğrudur?
	\begin{tasks}
          \task Diş dibinde yorulma çatlağı gözlenmektedir \correct
          \task Diş dibinde içkin bir süreksizlik gözlenmektedir
          \task Diş dibinde penetrant muayenesi sonrası ilgisiz belirti oluşacaktır
          \task Diş dibinde korozyonlu gerilim çatlağı gözlenmektedir
	\end{tasks}
\end{question}
\begin{solution}
	\correct
\end{solution}

\begin{question}[subtitle=]
  Aşağıdakilerden hangisi imalat aşamasında ortaya çıkar?
  \begin{tasks}
    \task Yorulma çatlağı
    \task Gerilimli korozyon çatlağı
    \task Katlanma \correct
    \task Aşınma sebepli çatlaklar
  \end{tasks}
\end{question}
\begin{solution}
	\correct
\end{solution}

\begin{question}[subtitle=]
  Dövme malzemelerde oluşacak damar, penetrant muayenesi sonucunda ne tür bir belirti oluşturacaktır
  \begin{tasks}
    \task Çizgisel \correct
    \task Yuvarlak
    \task Çizgisel veya Yuvarlak olabilir
    \task Damar hatası penetrant muayenesi ile tespit edilemez
  \end{tasks}
\end{question}
\begin{solution}
	\correct
\end{solution}

\begin{question}[subtitle=]
  Katmer(Laminasyon) hatasına aşağıdakilerden hangisinde sık rastlanılır?
  \begin{tasks}
    \task Döküm parçalarda
    \task Kaynaklı birleştirmelerde
    \task Haddelenmiş parçalarda \correct
    \task Dövme parçalarda
  \end{tasks}
\end{question}
\begin{solution}
	\correct
\end{solution}


