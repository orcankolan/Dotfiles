\begin{question}[subtitle=]
Sıvı Penetrant testi aşağıdakilerden hangi temel hareket prensibi ile çalışır? 
	\begin{tasks}
          \task Kırınım
          \task Difüzyon 
          \task Kapiler hareket \correct
          \task Ozmos
	\end{tasks}
\end{question}
\begin{solution}
	\correct
\end{solution}

\begin{question}[subtitle=]
Aşağıdakilerden hangisi penetrant muayenesi için uygun değildir? 
	\begin{tasks}
          \task Seramik yalıtıcılar
          \task Tek kristalli tribün bıçakları
          \task Titanyum bağlantı elemanı
          \task Örgü yapılı taşıyıcı çelik halatlar \correct
	\end{tasks}
\end{question}
\begin{solution}
	\correct
\end{solution}

\begin{question}[subtitle=]
Aşağıdakilerden hangisinin su bazlı ve çözücü bazlı çeşitleri vardır? 
	\begin{tasks}
          \task Penetrant
          \task Geliştirici \correct
          \task Emülsiyon yapıcı
          \task b ve c şıkları
	\end{tasks}
\end{question}
\begin{solution}
	\correct
\end{solution}

\begin{question}[subtitle=]
Boya penetrant kullanılırken arka plan kontrastı nasıl elde edilir? 
	\begin{tasks}
          \task Muayeneden önce yüzeye beyaz boya uygulayarak
          \task Fazla penetrantın temizlenmesinden sonra beyaz boya uygulayarak
          \task Tiksotropik penetrant kullanarak
          \task Geliştirici uygulayarak \correct
	\end{tasks}
\end{question}
\begin{solution}
	\correct
\end{solution}

\begin{question}[subtitle=]
Penetrant muayenesinin son aşaması olan son temizlik neden gereklidir? 
	\begin{tasks}
          \task Temizlenmemiş penetrant korozyona sebep olabilir
          \task Muayene sonrası kalıntılar kaplama ve boyama uygulamalarını olumsuz etkileyebilir
          \task Son temizlik mutlaka uygulanması gereken bir aşama değildir
          \task a ve b şıkları \correct
	\end{tasks}
\end{question}
\begin{solution}
	\correct
\end{solution}
