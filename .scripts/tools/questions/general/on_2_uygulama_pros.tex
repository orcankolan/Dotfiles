\begin{question}[subtitle=]
Penetrant muayenesinde Ön Temizlik aşaması ile aşağıdakilerden hangisi söylenemez?
	\begin{tasks}
          \task Kimyasal veya mekanik olarak yapılabilir 
          \task Zımparalama gibi metotlardan kaçınılmalıdır 
          \task Ön temizlik sonrası yüzeyin kuru kalması gerekmez \correct
          \task Çözücü sıvılar ile sadece organik kalıntılar giderilebilir
	\end{tasks}
\end{question}
\begin{solution}
	\correct
\end{solution}

\begin{question}[subtitle=]
Penetrant muayenesinden önce asit ile dağlama metodu için aşağıdakilerden hangisi söylenebilir?
	\begin{tasks}
          \task Penetrant muayenesinden önce asit ile dağlama yapılmamalıdır 
          \task Asit kalıntıları penetrantın florışıma özelliğini arttır 
          \task Tüm asidik ortam işlem sonucunda nötralize edilmelidir \correct
          \task Bu metot organik kalıntıları gidermek için uygulanır
	\end{tasks}
\end{question}
\begin{solution}
	\correct
\end{solution}

\begin{question}[subtitle=]
Yeterli veya uygun yapılmayan ön temizlik hangisine sebep olabilir?
	\begin{tasks}
          \task Hiçbir belirtinin oluşmamasına 
          \task İlgisiz belirti oluşmasına 
          \task Yanıltıcı / Yanlış belirti oluşmasına \correct
          \task Olması gerekenden daha büyük belirtilerin oluşmasına
	\end{tasks}
\end{question}
\begin{solution}
	\correct
\end{solution}

\begin{question}[subtitle=]
Mor Ötesi ışık şiddeti birimi aşağıdakilerden hangisidir?
	\begin{tasks}
          \task \si[per-mode=symbol]{\watt\per\m^{2}} 
          \task \si[per-mode=symbol]{\micro\watt\per\centi\m^{2}}
          \task a veya b şıkkı \correct
          \task Lüks
	\end{tasks}
\end{question}
\begin{solution}
	\correct
\end{solution}

\begin{question}[subtitle=]
  	\begin{figure}[!htb]
		\centering
		\fbox{\includegraphics[width=0.35\textwidth]{analoguvlm}}
	\end{figure}

  Yukarıdaki resim ile ilgili aşağıdaki ifadelerden hangisi yanlıştır?
	\begin{tasks}
          \task Soldaki cihaz ile UV ışık şiddeti ölçülmektedir
          \task Sağdaki cihaz ile beyaz ışık aydınlanması ölçülmektedir
          \task Sağdaki cihaz ile ölçülen değer florışıl penetrant uygulaması için uygundur 
          \task Sağdaki cihaz ile ölçülen değer boya penetrant uygulaması için uygundur \correct
	\end{tasks}
\end{question}
\begin{solution}
	\correct
\end{solution}

\begin{question}[subtitle=]
Aşağıdaki malzemelerden hangisi için en uzun penetrasyon süreleri gereklidir?
	\begin{tasks}
          \task Alüminyum 
          \task Östenitik alaşımlar
          \task Nikel bazlı östenitik alaşımlar \correct
          \task Titanyum
	\end{tasks}
\end{question}
\begin{solution}
	\correct
\end{solution}

\begin{question}[subtitle=]
  	\begin{figure}[!htb]
		\centering
		\fbox{\includegraphics[width=0.35\textwidth]{HydrophilicEmulsifier}}
	\end{figure}

  Yukarıdaki şema hangi metodun aşamalarını anlatmaktadır ?
	\begin{tasks}
          \task Penetrasyon sürecini
          \task Öz temizlik işlemini
          \task Hidrofilik emülsiyonlaştırma ile ara temizliği \correct
          \task Lipofilik emülsiyonlaştırma ile ara temizliği için uygundur
	\end{tasks}
\end{question}
\begin{solution}
	\correct
\end{solution}

\begin{question}[subtitle=]
Aşağıdakilerden hangisi İlgisiz Belirti sınıfına girer?
	\begin{tasks}
          \task Pres izlerinin ortaya çıkardığı belirtiler \correct
          \task Yüzeydeki fazla penetrantın iyi temizlenememesinden kaynaklanan arka-plan belirtileri
          \task Mikro çekinti boşluklarının sebep olduğu belirtiler
          \task Gözenek belirtileri
	\end{tasks}
\end{question}
\begin{solution}
	\correct
\end{solution}

\begin{question}[subtitle=]
Büyüklükleri, önceden belirlenmiş standartlardaki limitleri aşan büyüklüğe sahip bir süreksizliğin belirtisi nasıl adlandırılır?
	\begin{tasks}
          \task Geçerli belirti
          \task Hata \correct
          \task Yanlış Belirti
          \task Yukarıdakilerin hepsi
	\end{tasks}
\end{question}
\begin{solution}
	\correct
\end{solution}

\begin{question}[subtitle=]
  	\begin{figure}[!htb]
		\centering
		\fbox{\includegraphics[width=0.35\textwidth]{FalseIndications}}
	\end{figure}

Yukarıdaki resimde hangi tür belirtiler işaret edilmektedir?
	\begin{tasks}
          \task İlgili Belirtiler
          \task İlgisiz Belirtiler
          \task Hatalar
          \task Yanlış Belirtiler \correct
	\end{tasks}
\end{question}
\begin{solution}
	\correct
\end{solution}

\begin{question}[subtitle=]
Ara Yıkama aşaması ile ilgili aşağıdaki ifadelerden hangisi doğrudur?
	\begin{tasks}
          \task Su ile ara yıkamada yıkama 90 derece ile yapılmalıdır
          \task Lipofilik emülsiyonlar fırça ile tatbik edilebilir
          \task Hidrofilik emülsiyonlaştırmadan önce su ile mekanik temizlik yapılabilir \correct
          \task Yukarıdakilerin hepsi doğrudur
	\end{tasks}
\end{question}
\begin{solution}
	\correct
\end{solution}
