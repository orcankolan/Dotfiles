\begin{question}[subtitle=]
Aşağıdakilerden hangisi veya hangileri Yüzey Enerjisi (Yüzey Gerilimi) birimi olarak kullanılabilir?
	\begin{tasks}
          \task \si[per-mode=symbol]{\mN\per\mm^{2}}
          \task mJ/mm 
          \task mN/m \correct
          \task a ve c şıkları 
	\end{tasks}
\end{question}
\begin{solution}
	\correct
\end{solution}

\begin{question}[subtitle=]
Aşağıdakilerden hangisi sıvılar için doğru bir ifadedir?
	\begin{tasks}
          \task Sıcaklık arttıkça yüzey enerjileri artar
          \task Sıcaklık arttıkça yüzey enerjileri azalır \correct
          \task Sıcaklık azaldıkça yüzey enerjileri sabit kalır
          \task Sıcaklık azaldıkça yüzey enerjileri artabilir veya azalabilir
	\end{tasks}
\end{question}
\begin{solution}
	\correct
\end{solution}

\begin{question}[subtitle=]
Penetrant Sıvılarının iyi ıslatma özelliğine sahip olması gerekmektedir. Islatmanın ölçüsü olarak aşağıdakilerden hangisi gözlenir?
	\begin{tasks}
          \task Katı / Sıvı yüzey enerjisi
          \task Sıvı / Hava yüzey enerjisi
          \task Islatma açısı \correct
          \task Kapiler basınç
	\end{tasks}
\end{question}
\begin{solution}
	\correct
\end{solution}

\begin{question}[subtitle=]
Penetrant Sıvılarının kapiler basıncı için aşağıdaki ifadelerden hangisi yanlıştır?
	\begin{tasks}
          \task Düşük Olmalıdır \correct
          \task Sıvının yüzey gerilimine bağlıdır
          \task Islatma açısı azaldıkça artar
          \task Süreksizliğin yüzeye açıklığının genişliği ile ters orantılıdır.
	\end{tasks}
\end{question}
\begin{solution}
	\correct
\end{solution}

\begin{question}[subtitle=]
Penetrant Sıvılarının viskozitesi, pratikte neyi etkilemektedir ?
	\begin{tasks}
          \task Kapiler basıncı
          \task Penetrasyon hızını
          \task Penetrasyon sürelerini
          \task b ve c şıkları \correct
	\end{tasks}
\end{question}
\begin{solution}
	\correct
\end{solution}

\begin{question}[subtitle=]
Penetrant Sıvılarının  Sıcaklık Dayanıklılığı terimi aşağıdakilerden hangisi için kullanılır?
	\begin{tasks}
          \task Penetrantların sıcaklık karşısında ayrışmaya karşı gösterdiği direnç \correct
          \task Penetrantların yüksek sıcaklıklardaki florışıma özelliklerinin dayanıklılığı
          \task Penetrantların yüksek sıcaklıklardaki uçucu bileşenlerinin dayanıklılığı
          \task Penetrantların düşük sıcaklıklarda donmaya karşı direnci
	\end{tasks}
\end{question}
\begin{solution}
	\correct
\end{solution}


\begin{question}[subtitle=]
Çok yüksek sıcaklığın, penetrant sıvılarına etkisi aşağıdakilerden hangisidir?
	\begin{tasks}
          \task Yüksek sıcaklık ile penetrantın uçucu bileşen sıvıları azalacağından muayeneye olumlu etkisi vardır
          \task Yüksek sıcaklık ile florışıma özelliği artacağından muayeneye olumlu etkisi vardır
          \task Yüksek sıcaklık ile florışıma özelliği azalacağından muayeneye olumsuz etkisi vardır \correct
          \task Herhangi bir etkisi olmayacaktır
	\end{tasks}
\end{question}
\begin{solution}
	\correct
\end{solution}

\begin{question}[subtitle=]
Aşağıdakilerden hangisi penetrantın kapiler basıncını etkileyen etmenlerden biri değildir?
	\begin{tasks}
          \task Sıvı – Hava yüzeye enerjisi
          \task Penetrant sıvısı viskozitesi
          \task Süreksizliğin derinliği
          \task b ve c şıkları \correct
	\end{tasks}
\end{question}
\begin{solution}
	\correct
\end{solution}

\begin{question}[subtitle=]
Aşağıdakilerden hangisi penetrant sıvılarının özelliklerinden biri değildir?
	\begin{tasks}
          \task Florışıl Penetrant sıvısı içerisinde bulunan boya çözelti halindedir
          \task Penetrant sıvıları dağıtıcı katkı maddeleri içerirler
          \task Penetrant sıvılarının yüzey enerjileri yüksektir \correct
          \task b ve c şıkları
	\end{tasks}
\end{question}
\begin{solution}
	\correct
\end{solution}

\begin{question}[subtitle=]
Aşağıdaki elektromanyetik dalga boylarından hangisi elektromanyetik dalga spektrumunda morötesi ışık bandındadır?
	\begin{tasks}
          \task 300 nm
          \task 100 nm
          \task a ve b şıkkı \correct
          \task Hiçbiri
	\end{tasks}
\end{question}
\begin{solution}
	\correct
\end{solution}

\begin{question}[subtitle=]
Aşağıdaki ifadelerden hangisi florışıl penetrantlar için doğrudur?
	\begin{tasks}
          \task Boya konsantrasyonu arttıkça florışıma parlaklığı azalır
          \task Florışıma özelliklerinin sıcaklığa direncine Sıcaklık Dayanıklılığı denir
          \task Penetrantın, farklı sıcaklıklarda, boyasının çözeltiden ayrışma direncine Termal Dayanıklılık denir
          \task Hiçbiri \correct
	\end{tasks}
\end{question}
\begin{solution}
	\correct
\end{solution}

\begin{question}[subtitle=]
Aşağıdaki ifadelerden hangisi florışıl penetrantlar için doğrudur?
	\begin{tasks}
          \task Uzun süre UV Işınıma maruz kaldıklarında florışıma parlaklıkları azalır
          \task Penetrant filmi kalınlığının florışıma miktarına bir etkisi yoktur
          \task Florışıl penetrantlarda farklı soğurma özellikleri içeren boyalar bulunur
          \task Yukarıdakilerin Hepsi \correct
	\end{tasks}
\end{question}
\begin{solution}
	\correct
\end{solution}

\begin{question}[subtitle=]
Florışıl penetrantların florışıma parlaklığı hangi cihaz ile ölçülür?
	\begin{tasks}
          \task Refraktometre
          \task Hidrometre
          \task Florumetre \correct
          \task UV Metre
	\end{tasks}
\end{question}
\begin{solution}
	\correct
\end{solution}

\begin{question}[subtitle=]
Aşağıdakilerden hangisi çözücüler için doğru bir ifadedir?
	\begin{tasks}
          \task Çözücüler sadece sonradan emülsiyon halinde bulunan penetrantları temizleyebilirler
          \task Çözücüler ara yıkamada, yüzeye doğrudan sıkılarak tatbik edilebilir
          \task Çözücüler bez ile uygulanır ve tek yönlü silme işlemi ile tatbik edilir \correct
          \task a ve c şıkları
	\end{tasks}
\end{question}
\begin{solution}
	\correct
\end{solution}

\begin{question}[subtitle=]
Aşağıdakilerden hangisi Lipofilik Emülsiyon Yapıcı özelliklerinden biri değildir?
	\begin{tasks}
          \task Su toleransları limitsizdir \correct
          \task Doğrudan kullanıma hazır halde satılırlar
          \task Difüzyon mekanizması ile penetranta etki ederler
          \task Viskoziteleri yıkanabilme özelliklerini etkiler
	\end{tasks}
\end{question}
\begin{solution}
	\correct
\end{solution}

\begin{question}[subtitle=]
Aşağıdakilerden hangisi hidrofilik emülsiyon yapıcıların özelliklerinden biri değildir?
	\begin{tasks}
          \task Tatbikinden önce su ile ön temizlik yapılabilir
          \task Bu tür emülsiyon yapıcılar, lipofilik türünün aksine fırça ile tatbik edilebilir \correct
          \task Penetranta deterjan etkisi ile etki ederler
          \task Konsantrasyon halinde satılırlar
	\end{tasks}
\end{question}
\begin{solution}
	\correct
\end{solution}

\begin{question}[subtitle=]
Aşağıdakilerden hangisi çözücü bazlı geliştiricilerin özelliklerinden biri değildir?
	\begin{tasks}
          \task Çözücü etkileri vardır
          \task Sıvı emme ve tanecik yüzeyine tutunma etkileri vardır
          \task Tanecikleri yüksek öz kütleli olmalıdır \correct
          \task Florışıl penetrantların ışıma parlaklığını arttırırlar
	\end{tasks}
\end{question}
\begin{solution}
	\correct
\end{solution}

\begin{question}[subtitle=]
Aşağıdakilerden hangisi su bazlı geliştiricilerin sağlaması gereken özelliklerinden biri değildir?
	\begin{tasks}
          \task Parçacık tane boyutu çok çeşitli olmalıdır
          \task Tatbik öncesi kurutma gereklidir
          \task Korozyon önleyici katkı maddeleri içermelidirler
          \task a ve b şıkları \correct
	\end{tasks}
\end{question}
\begin{solution}
	\correct
\end{solution}

\begin{question}[subtitle=]
Aşağıdakilerden hangisi kuru toz geliştiriciler için yanlış bir ifadedir?
	\begin{tasks}
          \task Hem boya hem florışıl penetrantlar ile kullanılabilirler \correct
          \task UV Işımaya duyarsız olmalıdırlar
          \task Kimyasal tepkimeye girmemelidir
          \task Yukarıdakilerin hepsi
	\end{tasks}
\end{question}
\begin{solution}
	\correct
\end{solution}
