\begin{question}[subtitle=]
Süreksizlikleri Yorumlama ile bir muayene personeli, aşağıdakilerden hangi beceriyi sergilemektedir?
	\begin{tasks}
          \task Süreksizliklerin boyutlarının tespit edebilme
          \task Süreksizliklerin neden olabileceği hasarları tespit edebilme
          \task Süreksizlikleri meydana getirebilecek sebepleri tespit edebilme \correct
          \task Süreksizlikler için kabul veya ret kararı verebilme
	\end{tasks}
\end{question}
\begin{solution}
	\correct
\end{solution}

\begin{question}[subtitle=]
Bir dişlide oluşabilecek yorulma çatlağı ne tür bir süreksizlik sınıfına girmektedir?
	\begin{tasks}
          \task Proses süreksizliği
          \task Servis süreksizliği \correct
          \task İçkin süreksizlik
          \task Üretim süreksizliği
	\end{tasks}
\end{question}
\begin{solution}
	\correct
\end{solution}

\begin{question}[subtitle=]
Sıvı Penetrant Muayenesi sonucu ortaya çıkan çizgisel bir belirti, verilen muayene talimatlarındaki azami şartların altında bir boyuta sahip ise bu belirti nasıl adlandırılabilir?
	\begin{tasks}
          \task Hata
          \task İlgisiz belirti
          \task Süreksizlik \correct
          \task a ve c şıkları
	\end{tasks}
\end{question}
\begin{solution}
	\correct
\end{solution}

\begin{question}[subtitle=]
Aşağıdaki metal şekillendirme işlemlerinden hangisinde yuvarlak belirtilere rastlanmaz?
	\begin{tasks}
          \task Döküm
          \task Dövme \correct
          \task Kaynak
          \task b ve c şıkları
	\end{tasks}
\end{question}
\begin{solution}
	\correct
\end{solution}

\begin{question}[subtitle=]
Aşağıdakilerden hangisi hem döküm hem de kaynak işlemi sonucunda penetrant muayenesi ile görülebilen bir süreksizlik türü olabilir?
	\begin{tasks}
          \task Yetersiz ergime
          \task Katmer
          \task Gözenek \correct
          \task a ve c şıkları
	\end{tasks}
\end{question}
\begin{solution}
	\correct
\end{solution}

\begin{question}[subtitle=]
Aşağıdakilerden hangisi hem döküm hem de kaynak işlemi sonucunda görülebilen bir süreksizlik türü değildir?
	\begin{tasks}
          \task Yanma oluğu \correct
          \task Çatlak
          \task Gözenek
          \task Hiçbiri
	\end{tasks}
\end{question}
\begin{solution}
	\correct
\end{solution}

\begin{question}[subtitle=]
Penetrant muayenesi sonucu ince ve çizgisel bir belirtiye aşağıdakilerden hangisi sebep olmuş olabilir?
	\begin{tasks}
          \task Sıcak çatlak
          \task Yanma oluğu
          \task Gözenek
          \task Katlanma \correct
	\end{tasks}
\end{question}
\begin{solution}
	\correct
\end{solution}

\begin{question}[subtitle=]
Büyüklükleri, değerlendirme standartlarında belirtilen limitleri aşan belirtiler nasıl adlandırılırlar?
	\begin{tasks}
          \task Geçerli belirti
          \task Yanlış belirti
          \task Hata \correct
          \task Yukardakilerin hepsi
	\end{tasks}
\end{question}
\begin{solution}
	\correct
\end{solution}

\begin{question}[subtitle=]
Sıvı Penetrant Muayenesi ile ortaya çıkan çizgisel bir belirtiden edilen bilgi ile aşağıdakilerden hangisi söylenemez?
	\begin{tasks}
          \task Penetrant belirtisinin uzunluğu, süreksizliklerin gerçek uzunluğunu temsil etmez
          \task Belirtinin kalınlığı ile süreksizliğin gerçek derinliği bulunabilir \correct
          \task Belirtinin konumu, ve şekli belirtiye sebep olan sebepleri ortaya çıkarabilir
          \task Yukardakilerin hiçbiri
	\end{tasks}
\end{question}
\begin{solution}
	\correct
\end{solution}
