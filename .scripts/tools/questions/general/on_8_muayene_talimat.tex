\begin{question}[subtitle=]
Aşağıdakilerden hangisi bir muayene talimatında mutlaka olması gereken hususlardan biri değildir?
	\begin{tasks}
          \task Raporlandırma ve dokümantasyon
          \task Muayene parçası boyutları ve muayene kapsamı
          \task Muayene tekniği ve tekrar muayene koşulları
          \task Muayene personeli isimleri \correct
	\end{tasks}
\end{question}
\begin{solution}
	\correct
\end{solution}

\begin{question}[subtitle=]
Muayene Talimatı hazırlanırken, Belirtilerin Değerlendirilmesi bölümü ile ilgili aşağıdaki ifadelerden hangisi yalnıştır?
	\begin{tasks}
          \task Muayene sonucu bulunan belirtilerin değerlendirilmesi için bir standart kullanılıyorsa, bu standardın kriterlerinin dışında farklı kabul/ret kriterleri talimatı hazırlayan kurumun ihtiyaçları doğrultusunda talimata konabilir
          \task Belirtilerin değerlendirilmesi bölümü talimatın mutlaka ihtiva etmesi gereken hususlarından biridir
          \task Belirtilerin değerlendirilmesi ve kabul/ret gibi imza yetkili kararların alınması, eğer tüm kriterler açık bir şekilde ifade edildi ise 1. Seviye personel tarafından da gerçekleştirilebilir \correct
          \task Yukarıdakilerin hepsi doğrudur
	\end{tasks}
\end{question}
\begin{solution}
	\correct
\end{solution}

\begin{question}[subtitle=]
Yüzeyi taşlanmamış bir alın kaynak dikişinin penetrant muayenesi için aşağıdaki Muayene Talimatı ana başlıklarından hangisi, doğru bir değerlendirme için en fazla öneme sahiptir ve detaylı bir şekilde açıklanmalıdır?
	\begin{tasks}
          \task Muayeneyi değerlendirecek 2. Seviye personel sayısı
          \task Muayenenin yapılması gereken mekan ve zaman
          \task Muayene öncesi yüzey hazırlığı \correct
          \task Son temizlik
	\end{tasks}
\end{question}
\begin{solution}
	\correct
\end{solution}
