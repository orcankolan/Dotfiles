\begin{question}[subtitle=]
Aşağıdakilerden hangisi UV ışınım kaynaklarının kullanımı ile ilgili doğru bir ifadedir?
	\begin{tasks}
          \task Muayene personeli 380 nm’den küçük dalga boylu UV ışınıma maruz kalmamalıdır
          \task Penetrant muayenesinde UV-B ve UV-C ışınım kaynakları kullanılabilir
          \task UV - A türü ışınımı en yüksek enerji seviyeli UV ışınımı olduğu içi, bu tür ışınımlar penetrant muayenesinde kullanılmaz
          \task Yukarıdakilerin tümü yanlış ifadedir \correct
	\end{tasks}
\end{question}
\begin{solution}
	\correct
\end{solution}

\begin{question}[subtitle=]
Hassasiyet Seviyesi ile ilgili aşağıdakilerden hangisi yalnıştır?
	\begin{tasks}
          \task Boya penetrantlar için sadece 2 adet seviye mevcuttur.
          \task Seviyenin belirlenmesi için, hassasiyet seviyesi belirlenecek ürün ve referans blok kullanılması yeterlidir \correct
          \task Penetrantlara uygulanabildiği gibi ara yıkayıcı ve geliştiriciler için de seviyeler belirlenebilir
          \task Yukarıdakilerin tümü yanlış ifadedir
	\end{tasks}
\end{question}
\begin{solution}
	\correct
\end{solution}

\begin{question}[subtitle=]
  	\begin{figure}[!htb]
		\centering
		\fbox{\includegraphics[width=0.35\textwidth]{refblok2}}
	\end{figure}

Yukarıdaki resimde penetrant muayenesi uygulanmış blok ile ilgili aşağıdakilerden hangisi doğrudur?
	\begin{tasks}
          \task Hassasiyet seviyesi 4 olarak okunmaktadır
          \task Bloğun üst yarısı florışıl penetrantın yıkanabilirliğini ölçmek için kullanılır \correct
          \task Tip 1 Referans Test Bloğu olarak adlandırılır
          \task Yukarıdakilerin tümü yanlış ifadedir
	\end{tasks}
\end{question}
\begin{solution}
	\correct
\end{solution}

\begin{question}[subtitle=]
  	\begin{figure}[!htb]
		\centering
		\fbox{\includegraphics[width=0.35\textwidth]{asmeblok}}
	\end{figure}

Yukarıdaki resimde penetrant muayenesi uygulanmış blok ile ilgili aşağıdakilerden hangisi doğrudur?
	\begin{tasks}
          \task ASME Blok olarak bilinir
          \task Bir boya penetrant sisteminin hassasiyetinin tespiti için kullanılmaktadır
          \task Her iki (sağ ve sol) tarafa da aynı penetrant uygulanmıştır
          \task a ve b şıkları doğrudur \correct
	\end{tasks}
\end{question}
\begin{solution}
	\correct
\end{solution}

\begin{question}[subtitle=]
  	\begin{figure}[!htb]
		\centering
		\fbox{\includegraphics[width=0.35\textwidth]{refproduct}}
	\end{figure}

Yukarıdaki resimde penetrant muayenesi uygulanmış blok ile ilgili aşağıdakilerden hangisi doğrudur?
	\begin{tasks}
          \task Tüm plakalardaki çatlaklar görülebildiği için kullanılan florışıl penetrant hassasiyeti 4. Seviye olmalıdır.
          \task Penetrantın hassasiyetini belirleyebilmek için çatlaklar sayılmalı ve görülebilen çatlak sayısı toplam gerçek çatlak sayısına bölünerek bir yüzde bulunmalıdır.
          \task Hassasiyetin belirlenebilmesi için aday ürüne 2 kere muayene yapılması gereklidir
          \task Yukarıdakilerin tümü yanlış ifadedir \correct
	\end{tasks}
\end{question}
\begin{solution}
	\correct
\end{solution}

\begin{question}[subtitle=]
Proses Kontrol Testleri ile ilgili aşağıdaki ifadelerden hangisi doğrudur?
	\begin{tasks}
          \task Genellikle sabit istasyon tipi birimler için uygulanan testlerdir \correct
          \task Üretici firma tarafından gerçekleştirilebilir
          \task UBu testlerde kullanılan referans blok Tip 1 Referans Test Blok’tur
          \task Yukarıdakilerin tümü yanlış ifadedir
	\end{tasks}
\end{question}
\begin{solution}
	\correct
\end{solution}
