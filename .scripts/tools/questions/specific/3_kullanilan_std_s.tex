\begin{question}[subtitle=]
  Penetrant muayenesinde kullanılan malzemelere yapılan testlerde tanımlanan ``Tip testi''
  aşağıdakilerden hangisi tarafından gerçekleştirilir?
	\begin{tasks}
          \task Akredite laboratuvarlar tarafından \correct
          \task Üretici firma tarafından
          \task Muayene personeli tarafından
          \task a ve c şıkları
	\end{tasks}
\end{question}
\begin{solution}
	\correct
\end{solution}

\begin{question}[subtitle=]
  Hassasiyet seviyeleri için aşağıdaki ifadelerden hangisi yalnıştır?
	\begin{tasks}
          \task Florışıl penetrantlar için 5 adet seviye mevcuttur
          \task Referans ve aday ürün arasındaki kıyaslama yöntemi ile tespit edilir
          \task Muayene personeli tarafından gerçekleştirilmez
          \task Sadece ``Tip testinde'' gerçekleştirilir \correct
	\end{tasks}
\end{question}
\begin{solution}
	\correct
\end{solution}

\begin{question}[subtitle=]
Tip 2 Referans Blok Blok hangi amaçla kullanılabilir?
	\begin{tasks}
          \task Hassasiyet ölçümünde
          \task Penetrantların yıkanabilirliğinin kontrolünde
          \task Proses kontrol testlerinde, sistem performansın kontrolünde
          \task b ve c şıkları \correct
	\end{tasks}
\end{question}
\begin{solution}
	\correct
\end{solution}

\begin{question}[subtitle=]
Tip 1 Referans Blok için aşağıdaki ifadelerden hangisi yanlıştır?
	\begin{tasks}
          \task Hassasiyet seviyelerinin belirlenmesinde kullanılır
          \task Krom-nikel kaplı levhalardan oluşur
          \task Proses kontrol; Sistem performansın kontrolünde kullanılır \correct
          \task b ve c şıkları
	\end{tasks}
\end{question}
\begin{solution}
	\correct
\end{solution}

\begin{question}[subtitle=]
    	\begin{figure}[!htb]
		\centering
		\fbox{\includegraphics[width=0.75\textwidth]{refblok2-2}}
	\end{figure}

        Yukarıdaki resimde, proses kontrol testleri kapsamında yapılmış bir sistem performans deneyinin
        sonucu görülmektedir. Buna göre aşağıdaki ifadelerden hangisi(leri) doğrudur?
	\begin{tasks}
          \task Tip 2 referans test bloğu kullanılmıştır \correct
          \task Sistem performansı görünen belirti sayısı baz alındığında muayenelerin devamı
          için uygundur
          \task Kullanılan penetrantın yıkanabilirliği uygundur
          \task a ve b şıkları
	\end{tasks}
\end{question}
\begin{solution}
	\correct
\end{solution}

\begin{question}[subtitle=]
  Bir penetrant muayenesinde Tip 2 penetrant kullanılmaktadır. Buna göre aşağıdaki ifadelerden
  hangisi doğrudur?
	\begin{tasks}
          \task Belirtiler diğer tipteki penetrantlar ile aynı hassasiyette görülebilir
          \task Mor ötesi ışık kaynağına gerek yoktur \correct
          \task Yeterli kontrast için daha az geliştirici tabakası gereklidir
          \task Yukarıdakilerin hepsi 
	\end{tasks}
\end{question}
\begin{solution}
	\correct
\end{solution}

\begin{question}[subtitle=]
  Aşağıdakilerden hangisi proses kontrol testlerinden biri değildir?
	\begin{tasks}
          \task Hava filtrelerinin testi
          \task Florışıma şiddeti testi
          \task Hassasiyet seviyesi testi \correct
          \task a ve c şıkları
	\end{tasks}
\end{question}
\begin{solution}
	\correct
\end{solution}

\begin{question}[subtitle= EN ISO 3452-1]
	EN ISO 3452-1 standardına göre minimum geliştirme süresi ne kadardır?
	\begin{tasks}
		\task 5 dakika 
		\task 30 dakika 
		\task 10 dakika \correct
		\task Herhangi bir kısıtlama yoktur 
	\end{tasks}
\end{question}
\begin{solution}
	\correct
\end{solution}

\begin{question}[subtitle= EN ISO 3452-1]
	EN ISO 3452-1 standardına göre penetrant muayenesi için özel durumlarda sıcaklık aralığı ne
olabilir?
	\begin{tasks}
		\task 5-50 \si{\degreeCelsius} \correct
		\task 10-30 \si{\degreeCelsius} 
		\task 10-70 \si{\degreeCelsius}  
		\task 10-50 \si{\degreeCelsius}
	\end{tasks}
\end{question}
\begin{solution}
	\correct
\end{solution}

\begin{question}[subtitle= EN ISO 3452-1]
	Florışıl penetrant testi için aşağıdaki inceleme koşullarından hangisi en iyidir?
	\begin{tasks}
		\task 50 lx, 10 \si{\watt\per\m^{2}}
		\task 350 lx, 10 \si{\watt\per\m^{2}}
		\task 10 lx, 50 \si{\watt\per\m^{2}} \correct
		\task 30 lx, 30 \si{\watt\per\m^{2}}
	\end{tasks}
\end{question}
\begin{solution}
	\correct
\end{solution}

\begin{question}[subtitle= EN ISO 3452-1]
EN ISO 3452-1'e göre Florışıl Boya penetrant ara temizleyicinin su bazlı emülsiyon yapıcı ve su bazlı
yaş geliştiricinin kullanıldığı ürün ailesinin kısa gösterimi aşağıdakilerden hangisidir?
	\begin{tasks}
		\task EN ISO 3452-1-IDb \correct
		\task EN ISO 3452-1-IIDb
		\task EN ISO 3452-1-IAb
		\task EN ISO 3452-1-IAc 
	\end{tasks}
\end{question}
\begin{solution}
	\correct
\end{solution}


\begin{question}[subtitle= EN ISO 3452-2]
  Penetrant sistemleri için tip testinde duyarlılık (hassasiyet) seviyesi hangi blokla ve kim tarafından
yapılmalıdır?
	\begin{tasks}
		\task İmalatçı tarafından Tip 2 Blok ile
		\task Bağımsız bir laboratuvar tarafından Tip 1 Blok ile \correct
		\task Muayene personeli tarafından Tip 1 Blok ile
		\task Bağımsız bir laboratuvar tarafından Tip 2 Blok ile
	\end{tasks}
\end{question}
\begin{solution}
	\correct
\end{solution}

\begin{question}[subtitle= EN ISO 3059]
  UV-A ışınım kaynakları için aşağıdakilerden hangisi yanlış bir ifadedir?
	\begin{tasks}
		\task Filtre camı sıçrayan muayene ortamı sıvısına karşı dayanıklı olmalıdır
		\task UV-A ışınım kaynağının maksimum anma şiddeti 360nm olmalıdır \correct
		\task UV-A lambadan 400mm ötede UV-A ışınım şiddeti en az 10 \si{\watt\per\m^{2}} olmalıdır
		\task UV-A lambadan 400mm ötede aydınlanma şiddeti en az 20 lx'ten az olmamalıdır
	\end{tasks}
\end{question}
\begin{solution}
	\correct
\end{solution}

\begin{question}[subtitle=]
    	\begin{figure}[!htb]
		\centering
		\fbox{\includegraphics[width=0.40\textwidth]{asmeblok}}
	\end{figure}

        Yukarıdaki resimde bulunan blok hangi isim ile anılır?
	\begin{tasks}
          \task ASME Referans Blok \correct
          \task JIS Blok
          \task Referans Blok Tip 1
          \task Referans Blok Tip 2
	\end{tasks}
\end{question}
\begin{solution}
	\correct
\end{solution}

\begin{question}[subtitle= EN ISO 23277]
  EN ISO 23277'ye göre belirtiler ile ilgili aşağıdakilerden hangisi doğrudur?
	\begin{tasks}
		\task Uzunluğu genişliğinin 2 katından büyük belirtiler çizgisel belirti olarak sınıflandırılır
		\task Komşu belirtiler aralarındaki mesafeye bağlı olarak izole veya grup olarak sınıflandırılır \correct
		\task Kabul Seviyesi 2 için, izin verilebilir en büyük doğrusal belirti boyutu 3mm'dir 
		\task Yukarıdakilerin hepsi doğrudur
	\end{tasks}
\end{question}
\begin{solution}
	\correct
\end{solution}

\begin{question}[subtitle= EN ISO 23277]
  EN ISO 23277'ye göre 3X Kabul Seviyesi için aşağıdakilerden hangisi doğrudur?
	\begin{tasks}
		\task Doğrusal belirtiler için izin verilebilir en büyük belirti boyutu 8mm'dir
		\task Doğrusal belirtiler için izin verilebilir en büyük belirti boyutu 1mm'dir
		\task Doğrusal belirtiler için izin verilebilir en büyük belirti boyutu 2mm'dir  \correct
		\task Yuvarlak belirtiler için izin verilebilir en büyük belirti boyutu 6mm'dir
	\end{tasks}
\end{question}
\begin{solution}
	\correct
\end{solution}

\begin{question}[subtitle= EN 10228-2]
  EN 10228 standardı ile ilgili aşağıdakilerden hangisi(leri) yanlıştır?
	\begin{tasks}
		\task Dökümler için kabul seviyelerini belirler \correct
		\task Kabul muayenesi mümkünse parça müşteriye teslime hazır durumda iken yapılır
		\task Dört kalite sınıfı tanımlanmıştır
		\task b ve c şıkları yanlıştır
	\end{tasks}
\end{question}
\begin{solution}
	\correct
\end{solution}

\begin{question}[subtitle= EN 10228-2]
  EN 10228'e göre etkileşen belirtiler için aşağıdakilerden hangisi doğrudur?
	\begin{tasks}
		\task Çizgisel veya yuvarlak belirtiler etkileşen belirti oluşturabilirler
		\task İki belirti arasındaki mesafe, en büyük belirti boyutunun 5 katından küçük ise oluşur
		\task En az 3 belirti etkileşen belirti oluşturabilir
		\task Hiçbiri \correct
	\end{tasks}
\end{question}
\begin{solution}
	\correct
\end{solution}

\begin{question}[subtitle= EN 10228-2]
  EN 10228-2 ile belirtilerin değerlendirilmesi ile ilgili aşağıdakilerden hangisi doğrudur?
	\begin{tasks}
		\task Referans çerçeve içerisinde azami olması gereken belirti sayısı kontrol edilir
		\task Referans çerçeve içerisinde azami olması gereken toplam doğrusal belirti uzunluğu değeri kontrol edilir
		\task Referans çerçeve içerisinde bulunan yuvarlak belirtilerin sayısı hesaba katılmaz
		\task a ve b şıkları doğrudur \correct
	\end{tasks}
\end{question}
\begin{solution}
	\correct
\end{solution}

\begin{question}[subtitle= EN 1371-1]
  EN 1371-1 ile ilgili aşağıdaki ifadelerden hangisi doğrudur?
	\begin{tasks}
		\task Bakır/bronz alaşımlara uygulanamaz \correct
		\task Hizalanmış belirti sınıfı sadece doğrusal belirtiler için geçerlidir
		\task Doğrusal belirtiler izole ve grup halinde olmak üzere ikiye ayrılır
		\task a ve c şıkları doğrudur
	\end{tasks}
\end{question}
\begin{solution}
	\correct
\end{solution}

\begin{question}[subtitle= EN 1371-1]
  EN 1371-1'de belirtilen kalite sınıfları için aşağıdakilerden hangisi yanlıştır?
	\begin{tasks}
		\task Doğrusal olmayan belirtiler için 8 kalite sınıfı mevcuttur
		\task Doğrusal belirtiler için 7 kalite sınıfı mevcuttur
		\task Hizalanmış belirtiler doğrusal belirtiler için hazırlanan tablo ile değerlendirilir 
		\task Doğrusal olmayan belirtiler değerlendirilirken bulunduğu yüzeydeki kesit kalınlığına göre değerlendirlirler \correct
	\end{tasks}
\end{question}
\begin{solution}
	\correct
\end{solution}

\begin{question}[subtitle= EN 1371-1]
  32mm kesit kalınlığına sahip bir döküm yüzeyinin penetrant ile muayenesi sonucunda aşağıdaki belirtilerden hangisi Kalite Seviyesi 3 için standartta verilen koşulları sağlamamaktadır?
	\begin{tasks}
		\task Referans çerçeve içerisinde bulunacak 12 adet kayıt seviyesini geçen yuvarlak belirti
		\task Referans çerçeve içerisinde bulunacak 8mm'lik bir izole yuvarlak belirti
		\task Referans çerçeve içerisinde bulunacak 10mm'lik bir izole doğrusal belirti \correct
		\task Referans çerçeve içerisinde bulunacak 12mm'lik kümülatif toplama sahip doğrusal belirtiler
	\end{tasks}
\end{question}
\begin{solution}
	\correct
\end{solution}

\begin{question}[subtitle=]
Standartlar neyi tanımlamaz?
	\begin{tasks}
          \task Muayenenin nasıl yapılacağını
          \task Muayene yapacak personelin yeterliliğini
          \task Kabul ret kriterlerini
          \task Muayenede hangi marka cihazın kullanılması gerektiğini \correct
	\end{tasks}
\end{question}
\begin{solution}
	\correct
\end{solution}


\begin{question}[subtitle=]
EN ISO 3452-1'e göre Florışıl Boya penetrant ara temizleyicinin su bazlı emülsiyon yapıcı ve su bazlı
yaş geliştiricinin kullanıldığı ürün ailesinin kısa gösterimi aşağıdakilerden hangisidir?
	\begin{tasks}
          \task EN ISO 3452-1-IDb
          \task EN ISO 3452-1-IIDb \correct
          \task EN ISO 3452-1-IAb
          \task EN ISO 3452-1-IAc
	\end{tasks}
\end{question}
\begin{solution}
	\correct
\end{solution}


\begin{question}[subtitle=]
  Aşağıdakilerden hangisi standart çeşitlerinden değildir ?
	\begin{tasks}
          \task Tahribatsız muayenelerin anlatıldığı genel NDT metot standartları
          \task Üretim metotlarına (kaynak-döküm vb.) has standartlar
          \task Endüstride kullanılan özel ekipmanlar için kullanılan standartlar 
          \task Özel firmaların kendi ürünlerinin nasıl muayene edileceğini tanımladığı özel standartlar \correct
	\end{tasks}
\end{question}
\begin{solution}
	\correct
\end{solution}

\begin{question}[subtitle=]
  Muayene sonucunda tespit edilen bir süreksizliğin hata olup olmadığı hangi dokümana göre kararlaştırılmalıdır?
	\begin{tasks}
          \task Müşteri şartnameleri 
          \task Standartlar
          \task Standart ya da müşteri talimatları uyarınca hazırlanan test talimatları
          \task Yukarıdaki tüm  dokümanlardan yararlanılabilinir \correct
	\end{tasks}
\end{question}
\begin{solution}
	\correct
\end{solution}


\begin{question}[subtitle=]
  Aşağıdaki dokümanlardan hangisi(leri) tahribatsız muayene doküman(lar)ıdır?
	\begin{tasks}
          \task Muayene raporları \correct 
          \task Müşteri ile yapılan anlaşma
          \task Kaynak yöntem formları  
          \task Hepsi
	\end{tasks}
\end{question}
\begin{solution}
	\correct
\end{solution}


\begin{question}[subtitle=]
  Aşağıdakilerden hangisi yanlış bir ifadedir?
	\begin{tasks}
          \task EN ISO 3452-2 Standardında Tip Testi tanımlanır 
          \task ASTM E165 Standardı Kabul/Ret kriterlerini verir \correct
          \task EN ISO 3452-3 ile Referans Bloklar tanımlanır
          \task Hepsi
	\end{tasks}
\end{question}
\begin{solution}
	\correct
\end{solution}


\begin{question}[subtitle=]
  Aşağıdakilerden hangisi(leri) döküm parçaların penetrant muayenesi için kullanılabilir?
	\begin{tasks}
          \task EN 10246-11
          \task EN 23277
          \task EN 1371-1 \correct
          \task EN 1289
	\end{tasks}
\end{question}
\begin{solution}
	\correct
\end{solution}
