
\begin{question}[subtitle=]
	Tahribatsız muayeneleri yapacak personel hakkında aşağıdakilerden hangisi doğrudur?
	
	\begin{tasks}
		\task	Cihazı kullanmayı bilen her hangi biri testleri yapabilir
		\task	Kalite kontrol bölümünde çalışan herhangi biri yapabilir
		\task	Muayene yapacağı metotta sertifikalı biri yapabilir \correct
		\task	Hepsi doğrudur

	\end{tasks}
\end{question}
\begin{solution}
	\correct
\end{solution}

\begin{question}[subtitle=]
	Dünyada tahribatsız muayene eğitimi ve sertifikasyonu nasıl yapılmaktadır?
	
	\begin{tasks}
		\task	Üniversitelerin tahribatsız muayene bölümlerinde eğitimler verilerek sertifikalandırma yapılmaktadır
		\task	Uluslararası kabul görmüş eğitim ve sertifikasyon sistemleri uyarınca yapılmaktadır \correct
		\task	Her işletme kendi bilgi ve tecrübesi dahilinde yapmaktadır
		\task	Cihaz üreticilerinin verdiği eğitim ve sertifikalandırma yeterli olmaktadır
	\end{tasks}
\end{question}
\begin{solution}
	\correct
\end{solution}

\begin{question}[subtitle=]
	Katılmış olduğunuz iNDTOKULU eğitimi sonucunda hangi sisteme göre sertifika verilmektedir?
	
	\begin{tasks}
		\task ISO 9712
		\task EN473
		\task SNT-TC-1A \correct
		\task Sınav öncesi talep edilen sertifikaya göre (sadece biri seçilebilir)

	\end{tasks}
\end{question}
\begin{solution}
	\correct
\end{solution}

\begin{question}[subtitle=]
	SNT-TC 1A sertifikalandırma sistemi için aşağıdakilerden hangisi doğrudur?
	
	\begin{tasks}
		\task Sektör bazında  sertifikalandırma yapma imkanı vardır
		\task İşletme ihtiyaçlarına odaklanan sertifikalandırma sistemidir
		\task Sertifika personelin firmada çalıştığı sürece geçerlidir
		\task Hepsi doğrudur \correct
		
	\end{tasks}
\end{question}
\begin{solution}
	\correct
\end{solution}

\begin{question}[subtitle=]
	Tahribatsız muayene eğitimine katılma koşulu olarak aşağıdakilerden hangisi doğrudur?
	
	\begin{tasks}
\task 18 yaşını doldurmuş olmak \correct
\task En az  2 yıllık üniversite mezunu olmak
\task Gözlük kullanmamak
\task Yukarıdaki koşullardan herhangi birini sağlamayanlar eğitime katılamazlar.

	\end{tasks}
\end{question}
\begin{solution}
	\correct
\end{solution}

\begin{question}[subtitle=]
	Seviye 1 sertifikasına sahip bir personelin aşağıdakilerden hangisine yetkisi yoktur?
	
	\begin{tasks}
		\task Cihazı çalıştırıp kullanma
		\task Parçayı muayene etmeye
		\task Bulguları kayıt etmeye
		\task Raporu imzalamaya \correct
	
	\end{tasks}
\end{question}
\begin{solution}
	\correct
\end{solution}

\begin{question}[subtitle=]
	Muayene prosedür ve talimatlarını kim hazırlamalıdır?
	
	\begin{tasks}
		\task Seviye 1 sertifikalı personel
		\task En az Seviye 2 sertifikalı personel \correct
		\task Kalite kontrol müdürü
		\task Şirket müdürü 
			
	\end{tasks}
\end{question}
\begin{solution}
	\correct
\end{solution}

\begin{question}[subtitle=]
	SNT-TC 1A  sisteminde sertifika almaya hak kazanmak için aşağıdakilerden hangisi doğrudur?
	
	\begin{tasks}
		\task 3 sınav ortalamasının 100 üzerinde 80 olması gerekmektedir
		\task Herhangi bir sınavdan en az 70 alınması gerekmektedir 
		\task a ve b şıklarındaki koşulların sağlanması gerekmektedir \correct
		\task Sadece ortalamanın 80 olması gerekmektedir

	\end{tasks}
\end{question}
\begin{solution}
	\correct
\end{solution}


\begin{question}[subtitle=]
	Sınav sonunda aşağıdaki sonuçları alan bir katılımcı için aşağıdakilerden hangisi doğrudur?\par
	Genel Bilgi Sınavı = 69,    Özel Bilgi Sınavı = 100,  Pratik Sınavı = 100 
		
	\begin{tasks}
		\task Sertifika almaya hak kazanmıştır
		\task 70 in altında notu olduğu için sertifika almaya hak kazanamamıştır \correct
		\task Ortalaması 80 in altına olduğu için sertifika almaya hak kazanamamıştır
		\task Ortalaması 70 in üstünde olduğu için sertifika almaya hak kazanmıştır

	\end{tasks}
\end{question}
\begin{solution}
	\correct
\end{solution}


\begin{question}[subtitle=]
	SNT-TC 1A  sisteminde sertifika süresi hakkında aşağıdakilerden hangisi doğrudur? 
	
	\begin{tasks}
		\task Sertifikalar süresizdir
		\task En az 5 yıldır
		\task En çok 5 yıldır \correct
		\task Sertifika her yıl uzatılmalıdır

	\end{tasks}
\end{question}
\begin{solution}
	\correct
\end{solution}