
\begin{question}[subtitle=]
Aşağıdakilerden hangisi(leri), tahribatsız muayene metodudur?

	\begin{tasks}
		\task Görsel muayene
		\task Ultrasonik muayene
		\task Radyografik muayene 
		\task Yukarıdakilerin hepsi \correct
	\end{tasks}
\end{question}
\begin{solution}
	\correct
\end{solution}

\begin{question}[subtitle=]
	Aşağıdakilerden hangisi(leri), tahribatsız muayene metodudur?
	
	\begin{tasks}
		\task Bükme testi
		\task Eğme testi
		\task Çekme testi
		\task Yukarıdakilerin hiçbiri \correct
	\end{tasks}
\end{question}
\begin{solution}
	\correct
\end{solution}

\begin{question}[subtitle=]
	Tahribatsız muayene tanımı için aşağıdakilerden hangisi doğrudur?
	
	\begin{tasks}
		\task Malzemenin işlevini yitirmesine yol açabilir
		\task Malzemenin boyutsal değişikliğine sebep olabilir
		\task Malzemeyi tahrip etmez \correct
		\task Malzemenin kimyasal değişikliğine sebep olabilir
	\end{tasks}
\end{question}
\begin{solution}
	\correct
\end{solution}

\begin{question}[subtitle=]
	Aşağıdaki muayene metotlarından hangisi, dünya genelinde en yaygın olarak kullanılan 6 muayene metodundan biri değildir?
	
	\begin{tasks}
		\task Radyografik Muayene
		\task Vibrasyon Testleri \correct
		\task Manyetik Parçacık Muayenesi
		\task Gözle muayene
	\end{tasks}
\end{question}
\begin{solution}
	\correct
\end{solution}

\begin{question}[subtitle=]
	Aşağıdaki muayene metotlarından hangisi, tek başına ayrı bir metot olmakla birlikte, genelde diğer metotları uygulama öncesi   tamamlayıcı metot  olarak da kullanılır?
	
	\begin{tasks}
		\task Gözle Muayene \correct
		\task Ultrasonik Muayene 
		\task Radyografik Muayene
		\task Manyetik Parçacık Muayenesi
	\end{tasks}
\end{question}
\begin{solution}
	\correct
\end{solution}

\begin{question}[subtitle=]
	En iyi tahribatsız muayene metodu hangisidir?
	
	\begin{tasks}
		\task Ultrasonik Muayene
		\task Radyografik Muayene
		\task Gözle Muayene
		\task Uygulama özeline en uygun muayene metodu hangisi ise	\correct	
	\end{tasks}
\end{question}
\begin{solution}
	\correct
\end{solution}

\begin{question}[subtitle=]
	Malzeme yüzeyindeki hataların tespiti için aşağıdaki muayene metotlarından hangisi(leri) kullanılabilir?
	
	\begin{tasks}
		\task Gözle Muayene 
		\task Manyetik Parçacık Muayenesi
		\task Sıvı Penetrant Muayenesi
		\task Hepsi kullanılabilir	\correct
	\end{tasks}
\end{question}
\begin{solution}
	\correct
\end{solution}

\begin{question}[subtitle=]
	Malzeme yüzeyi altındaki (içindeki) hacimsel hataların tespiti için aşağıdaki muayene metotlarından hangisi(leri) kullanılabilir?
	
	\begin{tasks}
		\task Gözle Muayene 
		\task Ultrasonik Muayene \correct
		\task Sıvı Penetrant Muayenesi
		\task Hepsi kullanılabilir
	\end{tasks}
\end{question}
\begin{solution}
	\correct
\end{solution}

\begin{question}[subtitle=]
	Malzeme içine yüksek enerjili radyoaktif  ışınların gönderilmesi prensibine dayanan muayene metodu hangisidir?
	
	\begin{tasks}
		\task Ultrasonik Muayene
		\task Radyografik Muayene \correct
		\task Manyetik Parçacık Muayenesi
		\task Girdap Akımları Muayenesi
	\end{tasks}
\end{question}
\begin{solution}
	\correct
\end{solution}

\begin{question}[subtitle=]
	Boyutsal kontrolde en yaygın olarak kullanılan muayene metodu hangisidir?
	
	\begin{tasks}
		\task Gözle Muayene \correct
		\task Ultrasonik Muayene
		\task Radyografik Muayene
		\task Girdap Akımları Muayenesi
	\end{tasks}
\end{question}
\begin{solution}
	\correct
\end{solution}